\documentclass[]{article}

\usepackage[a4paper, total={6in, 8in}]{geometry}

\usepackage[brazil]{babel}
\usepackage[utf8]{inputenc}

\usepackage{amsmath}
\usepackage{amsfonts}
\usepackage{amsthm}
\usepackage{listings}
\usepackage{hyperref}
\usepackage{graphics}
\usepackage{tikz}
\usepackage{mathtools}

\DeclarePairedDelimiter\ceil{\lceil}{\rceil}
\DeclarePairedDelimiter\floor{\lfloor}{\rfloor}

\begin{document}

\begin{center}
  \Large\textbf{Entrega 4}\\
  \large\textit{Daniel Brito}
\end{center}

12) Dado um conjunto de inteiros $A = \{a_1, a_2, ..., a_n\}$ e um inteiro t, o seguinte algoritmo decide se o somatório de algum subconjunto de $A$ é maior que $t$:

\begin{lstlisting}
import sys

def somatorioMaiorT(a, t):
    
    soma = 0
    l = sys.maxsize
    n = len(a)
 
    esq = 0
 
    for dir in range(n):
        soma += a[dir]
 
        while soma > t and esq <= dir:
            l = min(l, dir - esq + 1)

            soma -= a[esq]
            esq += 1
 
    return l != sys.maxsize
\end{lstlisting}

\newpage

13) Dado um conjunto de inteiros $A = \{a_1, a_2, ..., a_n\}$ e um inteiro t, o seguinte algoritmo decide se o somatório de algum subconjunto de $A$ é igual a $t$:

\begin{lstlisting}
def somatorioIgualT(a, n, soma):
    
  sub = ([[False for i in range(soma+1)] for i in range(n+1)])
  
  for i in range(n+1):
    sub[i][0] = True
    
  for i in range(1, soma+1):
    sub[0][i]= False
      
  for i in range(1, n+1):
    for j in range(1, soma+1):
        
      if j < a[i-1]:
        sub[i][j] = sub[i-1][j]
        
      if j >= a[i-1]:
        sub[i][j] = (sub[i-1][j] or sub[i-1][j - a[i-1]])
  
  return sub[n][soma]
\end{lstlisting}

\newpage

14) Dado um conjunto de inteiros $A = \{a_1, a_2, ..., a_n\}$, o seguinte algoritmo determina se é possível particioná-lo em duas partes, de maneira que o somatório dos elementos de cada parte é igual, retornando as partições ou -1, caso não seja possível:

\begin{lstlisting}
def particaoIgual(A):
  for i in range(len(A)):
    soma_esq = 0
        
    for j in range(i):
      soma_esq += A[j]
 
    soma_dir = 0
        
    for j in range(i, len(A)):
      soma_dir += A[j]
 
    if soma_esq == soma_dir:
      return [A[:i], A[i:]]
 
  return -1
\end{lstlisting}

\newpage

15) Dado um conjunto de inteiros $A = \{a_1, a_2, ..., a_n\}$, o seguinte algoritmo determina se é possível particioná-lo em duas partes, de maneira que a diferença absoluta entre o somatório de cada parte seja mínima:

\begin{lstlisting}
import sys

def findMin(A):
  soma = sum(A)
  n = len(A)

  dp = [[0 for i in range(soma+1)] 
      for j in range(n+1)]

  for i in range(n+1):
    dp[i][0] = True
    
  for j in range(1, soma+1):
    dp[0][j] = False
  
  for i in range(1, n+1):
    for j in range(1, soma+1):
      
      dp[i][j] = dp[i-1][j]
      
      if a[i-1] <= j:
        dp[i][j] |= dp[i-1][j-a[i-1]]
  
  dif = sys.maxsize

  for j in range(soma//2, -1, -1):
    if dp[n][j] == True:
      dif = soma - (2*j)
      break
      
  return dif
\end{lstlisting}

\newpage

16) Dado um conjunto de inteiros $A = \{a_1, a_2, ..., a_n\}$, \texttt{particionaTrio} particiona $A$ em três partes cujo somatório é igual entre si. Logo após, \texttt{verificaParticoes} verifica se foi possível realizar a divisão com base nesta restrição, imprimindo o devido resultado:

\begin{lstlisting}
def particionaTrio(A, pos1, pos2): 
  n = len(A); 

  prefixo = [0] * n; 
  soma = 0; 

  for i in range(n): 
    soma += A[i]; 
    prefixo[i] = soma; 

  sufixo = [0] * n; 
  soma = 0; 
    
  for i in range(n - 1, -1, -1): 
    soma += A[i]; 
    sufixo[i] = soma; 

  somaTotal = soma; 

  i = 0; 
  j = n - 1; 
    
  while i < j-1: 
    if prefixo[i] == somaTotal // 3: 
      pos1 = i; 

    if sufixo[j] == somaTotal // 3: 
      pos2 = j; 

    if pos1 != -1 and pos2 != -1: 
      if sufixo[pos1 + 1] - sufixo[pos2] == somaTotal // 3: 
        return [True, pos1, pos2]; 
      else: 
        return [False, pos1, pos2]; 

    if prefixo[i] < sufixo[j]: 
      i += 1; 
    else: 
      j -= 1; 

  return [False, pos1, pos2]; 




def verificaParticoes(A): 

  pos1 = -1; 
  pos2 = -1; 
    
  res = particionaTrio(A, pos1, pos2); 
    
  pos1 = res[1]; 
  pos2 = res[2];
    
  if res[0]:
    for i in range(pos1 + 1): 
      print(A[i], end = " ");
        
    print("")

    for i in range(pos1 + 1, pos2): 
      print(A[i], end = " "); 
            
    print("")

    for i in range(pos2, len(A)): 
      print(A[i], end = " "); 
  else: 
    print("Nao foi possivel particionar"); 
\end{lstlisting}

\newpage

17) Abaixo, o algoritmo de Bellman-Ford alterado de forma que o conteúdo do primeiro laço precise ser executado apenas $m+1$ vezes, mesmo que o valor de $m$ seja desconhecido:

\begin{lstlisting}
BELLMAN-FORD-ALT(G, w, s)

INITIALIZE-SINGLE-SOURCE(G, s)
changes = True
while changes == True:
    changes = FALSE
    for each edge (u, v) in G.E.
        RELAX-M(u, v, w)
        
RELAX-M(u, v, w)
if v.d > u.d + w(u, v)
    v.d = u.d + w(u, v)
    v.PI = u
    changes = True
\end{lstlisting}

\newpage

18) TO-DO

\begin{lstlisting}

\end{lstlisting}

\newpage

19) TO-DO

\newpage

20) TO-DO

\begin{lstlisting}

\end{lstlisting}

\end{document}