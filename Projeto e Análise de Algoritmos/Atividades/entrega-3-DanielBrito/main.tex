\documentclass[]{article}

\usepackage[a4paper, total={6in, 8in}]{geometry}

\usepackage[brazil]{babel}
\usepackage[utf8]{inputenc}

\usepackage{amsmath}
\usepackage{amsfonts}
\usepackage{amsthm}
\usepackage{listings}
\usepackage{hyperref}
\usepackage{graphics}
\usepackage{tikz}
\usepackage{mathtools}

\DeclarePairedDelimiter\ceil{\lceil}{\rceil}
\DeclarePairedDelimiter\floor{\lfloor}{\rfloor}

\begin{document}

\begin{center}
  \Large\textbf{Entrega 3}\\
  \large\textit{Daniel Brito}
\end{center}

1) [OPCIONAL] Em uma heap, sabemos que o filho esquerdo de um nó é indexado por $\floor*{\frac{n}{2}} + 1$, e também que heaps são árvores quase completas, ou seja, não é adicionado um filho à direita sem que antes seja adicionado o filho à esquerda. Assim, temos que:

\begin{align*}
    ESQ(\floor*{\frac{n}{2}} + 1) & = 2(\floor*{\frac{n}{2}} + 1) \\
                                       & > 2(\frac{n}{2} - 1) + 2 \\
                                       & = n - 2 + 2 \\
                                       & = n
\end{align*}

Uma vez que o elemento do índice da esquerda representa um número maior que a quantidade de elementos da heap, o nó não possui filhos, logo, é uma folha. O mesmo acontece com todos os nós com índices maiores.

\newpage

2) [OPCIONAL] Note que, para qualquer $n>0$, o número de folhas de uma árvore quase completa é $\ceil*{\frac{n}{2}}$.

Seja $n_h$ o número de nós em uma altura $h$. O \textit{upper bound} é válido para o caso base, uma vez que $n_0 = \ceil*{\frac{n}{2}}$ é exatamente o número de folhas em uma heap de tamanho $n$.

Passo indutivo: Suponha que é verdade para $h-1$. Assim, temos que provar que também é válido para $h$.

Note que se $n_{h-1}$ é par, cada nó de uma árvore de altura $h$ tem exatamente dois filhos, o que implica $n_h = \frac{n_{h-1}}{2} = \floor*{\frac{n_{h-1}}{2}}$.

Se $n_{h-1}$ é ímpar, um nó na altura $h$ tem um filho, e o restante tem dois filhos, o que implica $n_h = \floor*{\frac{n_{h-1}}{2}} + 1 = \ceil*{\frac{n_{h-1}}{2}}$.

Assim, temos que:

\begin{align*}
    n_h & = \ceil*{\frac{n_{h-1}}{2}} \\
        & \leq \ceil*{\frac{1}{2} \cdot \ceil*{\frac{n}{2^{(h-1)+1}}}} \\
        & = \ceil*{\frac{1}{2} \cdot \ceil*{\frac{n}{2^h}}} \\
        & = \ceil*{\frac{n}{2^{h+1}}}
\end{align*}

\newpage

3) [OPCIONAL] Abaixo, o algoritmo que determina os valores que representam o infixo de soma máxima em uma sequência de números:

\begin{lstlisting}
from sys import maxsize as MAX

def somaMaxima(a, n):
  maximo = - MAX
  somaAtual = 0
  inicio = 0
  fim = 0
  s = 0

  for i in range(n):
    somaAtual += a[i]

    if maximo < somaAtual:
      maximo = somaAtual
      inicio = s
      fim = i

    if somaAtual < 0:
      somaAtual = 0
      s = i + 1
            
  return a[inicio:fim+1]
\end{lstlisting}

\newpage

4) Abaixo, o algoritmo que determina em que hotéis deve-se parar, de forma que a soma das penalidades diárias seja minimizada:

\begin{lstlisting}
def hoteis(a):
  n = len(a)
  penalidades = [0]*n
  paradas = [0]*n
    
  for i in range(n):
    penalidades[i] = int(math.pow(200 - a[i], 2))
    paradas[i] = 0
        
    for j in range(i):
      if penalidades[j] + math.pow((200 - (a[i] - a[j])), 2) < penalidades[i]:
        penalidades[i] = int(penalidades[j] + math.pow(200 - (a[i] - a[j]), 2))
        paradas[i] = j + 1
                
  resultado = []
  index = len(paradas)-1
    
  while index >= 0:
    resultado.append(index + 1)
    index = paradas[index] - 1
    
  return resultado[::-1]
\end{lstlisting}

\newpage

5) Abaixo, o algoritmo que decide se uma dada string \texttt{s[1..n]} pode ser segmentada em uma sequência de palavras de um dicionário \texttt{d}:

\begin{lstlisting}
def consulta(d, s):
  if not s:
    return True
 
  for i in range(1, len(s) + 1):
    prefixo = s[:i]
 
    if prefixo in d and consulta(d, s[i:]):
      return True
        
  return False
\end{lstlisting}

\newpage

6) Abaixo, o algoritmo que encontra a subsequência palindrômica máxima de $a_1, a_2, ..., a_n$:

\begin{lstlisting}
def palindromo(a):
  dp = [[False for i in range(len(a))] for i in range(len(a))]
  
  for i in range(len(a)):
     dp[i][i] = True
     
  max = 1
  inicio = 0
  
  for i in range(2, len(a) + 1):
    for j in range(len(a)- i + 1):
      fim = j + i
        
      if i == 2:
        if a[j] == a[fim-1]:
          dp[j][fim-1] = True
          max = i
          inicio = j
      else:
        if a[j] == a[fim-1] and dp[j+1][fim-2]:
          dp[j][fim-1] = True
          max = i
          inicio = j

  return a[inicio:inicio+max]
\end{lstlisting}

\newpage

7) Abaixo, o algoritmo que encontra o maior comprimento de uma substring comum entre duas strings \texttt{x[1..m]} e \texttt{y[1..n]}:

\begin{lstlisting}
def comprimentoSubstring(x, y):
  m = len(x)
  n = len(y)
  
  comum = [[0 for i in range(n+1)] for j in range(m+1)]

  maiorComprimento = 0

  for i in range(m + 1):
    for j in range(n + 1):
      if i == 0 or j == 0:
        comum[i][j] = 0
      elif (x[i-1] == y[j-1]):
        comum[i][j] = comum[i-1][j-1] + 1
        maiorComprimento = max(maiorComprimento, comum[i][j])
      else:
        comum[i][j] = 0

  return maiorComprimento
\end{lstlisting}

\newpage

8) Abaixo, o algoritmo para calcular a probabilidade de se obter exatamente \texttt{k} caras quando cada uma das \texttt{n} moedas é arremessada uma vez:

\begin{lstlisting}
from math import log2 

def probabilidade(k, n):
  resposta = 0
  
  for i in range(k, n+1):
    exp = e[n] - e[i] - e[n-i] - n 
    resposta += pow(2.0, exp)
    
  return resposta
\end{lstlisting}

\vspace{0.5cm}

Para otimizar o cálculo do fatorial, é realizada uma pré-computação de \texttt{e[]}, utilizando o módulo \texttt{log2} da biblioteca \texttt{math}. Caso contrário, corre-se o risco de ocorrer overflow para valores muito altos:

\begin{lstlisting}
N = 1000001
e = [0] * N

for i in range(2, N):
  e[i] = log2(i) + e[i-1] 
\end{lstlisting}

\newpage

9) TO-DO

\newpage

10) Abaixo, o algoritmo para determinar o número mínimo de moedas necessárias para representar um certo valor \texttt{v}:

\begin{lstlisting}
def minMoedas(x, v):
  if v == 0:
    return 0
    
  if v < 0:
    # Caso nao seja possivel representar v:
    return float('INF')
    
  moedas = float('INF')
  n = len(x)

  for i in range(n):
    m = minMoedas(x, v - x[i])

    if m != float('INF'):
      moedas = min(moedas, m + 1)

  # Caso seja possivel representar v:
  return moedas
\end{lstlisting}

\newpage

11) TO-DO

\end{document}