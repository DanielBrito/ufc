\documentclass[]{article}

\usepackage[brazil]{babel}
\usepackage[utf8]{inputenc}
\usepackage{hyperref}
\usepackage{graphicx}
\usepackage{multirow}
\usepackage{rotating}
\usepackage{latexsym}
\usepackage{subfigure}
\usepackage{amsmath}
\usepackage{amsfonts}
\usepackage{amssymb}
\usepackage{amsthm}
\usepackage{float}
\usepackage{fancyhdr}
%\usepackage{breakurl}
\usepackage{tabularx}
\usepackage{indentfirst}
\usepackage{mathtools}
\usepackage{pdfpages}
\usepackage{threeparttable}
\usepackage{adjustbox}

\usepackage{pgfplots}
\pgfplotsset{width=7cm,compat=1.8}

\usepackage{tikz}
\usetikzlibrary{positioning}

\usepackage{caption}

\usepackage[portuguese, ruled, linesnumbered]{algorithm2e}

\usepackage{pdfpages}

%opening
\title{Lista de Exercícios - Projeto e Análise de Algoritmos}
\author{Luiz Alberto do Carmo Viana}

\begin{document}

\maketitle

\vspace{\baselineskip}

\textbf{Questão 1}

Mostre que, se um array de $n$ posições representa uma heap de $n$
elementos, então as folhas dessa heap estão nas posições do array de
índices
$\lfloor \frac{n}{2} \rfloor + 1, \lfloor \frac{n}{2} \rfloor + 2,
\dots, n$.

\vspace{\baselineskip}

\textbf{Questão 2}

Mostre que existem no máximo $\lceil \frac{n}{2^{h + 1}} \rceil$ nós
de altura $h$ em uma heap com $n$ elementos.

\vspace{\baselineskip}

\textbf{Questão 3}

Dada uma sequência de números $a_1, a_2, \dots, a_n$, um infixo dessa
sequência é uma subsequência da forma
$a_i, a_{i + 1}, a_{i + 2}, \dots, a_j$, para
$1 \leq i \leq j \leq n$.  Elabore um algoritmo para determinar o
infixo de soma máxima de uma sequência de números.

\vspace{\baselineskip}

\textbf{Questão 4}

No começo de uma viagem, você se encontra no km $0$.  No caminho,
existem $n$ hotéis, nos km's $a_1 < a_2 < \dots < a_n$.  Ao final de
um dia de viagem, você deve fazer uma parada em um desses hotéis.
Você deve parar no hotel do km $a_n$, já que esse é seu destino.
Idealmente, você gostaria de viajar 200 km por dia.  Sendo assim, se
você viaja $x$ km em um dia, há uma penalidade de $(200 - x)^2$ para
aquele dia de viagem.  Elabore um algoritmo que determina em que
hotéis você deve parar, de forma que a soma das penalidades diárias
seja minimizada.

\vspace{\baselineskip}

\textbf{Questão 5}

Dada uma string de $n$ caracteres $s[1 \dots n]$, gostaríamos de
decidir se $s$ pode ser ``quebrada'' em uma sequência de palavras de
um certo dicionário $d$.  Consultar o dicionário $d$, isto é,
determinar se uma certa substring contígua de $s$ está em $d$ custa
$O(1)$.  Desenvolva um algoritmo com custo de execução $O(n^2)$ para
resolver esse problema.

\pagebreak

\vspace{\baselineskip}

\textbf{Questão 6}

Dada uma sequência de números $a_1, a_2, \dots, a_n$,
$n \in \mathbb{N}$, dizemos que uma subsequência
$a_i, a_{i + 1}, \dots, a_j$ é palindrômica se
$a_i = a_j, a_{i + 1} = a_{j - 1}$ e assim por diante.  Desenvolva um
algoritmo com custo de execução $O(n^2)$ que encontra a subsequência
palindrômica máxima de $a_1, a_2, \dots, a_n$.

\vspace{\baselineskip}

\textbf{Questão 7}

Dadas duas strings $x[1 \dots m]$ e $y[1 \dots n]$, gostaríamos de
encostrar o comprimento de sua maior substring comum, isto é, o maior
valor de $k$ para o qual haja índices $i \in [m - k + 1]$ e
$j \in [n - k + 1]$ tais que
$x_ix_{i + 1} \dots x_{i + k - 1} = y_jy_{j + 1} \dots y_{j + k - 1}$.
Para tanto, desenvolva um algoritmo com custo de execução $O(mn)$.

\vspace{\baselineskip}

\textbf{Questão 8}

Tome $n$ moedas viciadas, com probabilidades
$p_1, p_2, \dots, p_n \in [0, 1]$ de resultarem em cara.  Qual a
probabilidade de se obter exatamente $k$ caras quando cada uma dessas
$n$ moedas é arremessada uma vez?  Desenvolva um algoritmo que
responda essa pergunta em tempo $O(n^2)$.

\vspace{\baselineskip}

\textbf{Questão 9}

Tome uma chapa retangular de aço, com dimensões $M \times N$.  Uma
unidade do produto $p_i$ gera um lucro $c_i$ quando vendida, e precisa
de exatamente uma chapa retangular de aço com dimensões
$x_i \times y_i$ para ser fabricada, $i \in [n]$.  Desenvolva um
algoritmo para determinar a melhor forma de cortar a chapa de aço
$M \times N$, isto é, que permita fabricar unidades dos produtos de
forma a maximizar o lucro.

\vspace{\baselineskip}

\textbf{Questão 10}

Tome uma quantidade ilimitada de moedas com valores
$x_1, x_2, \dots, x_n$.  Desenvolva um algoritmo que determina o
número mínimo de moedas necessárias para representar um certo valor
$v$, caso seja possível representar $v$.  Seu algoritmo deve ter custa
de tempo $O(nv)$.  Agora, resolva esse problema considerando que não
se pode usar moedas repetidas.

\vspace{\baselineskip}

\textbf{Questão 11}

Dado um grafo $G = (V, E)$, uma cobertura de vértices de $G$ é um
subconjunto $S \subseteq V$ tal que $S \cap e \neq \emptyset$, para
toda aresta $e \in E$.  Uma árvore é um grafo conexo e acíclico.  Dada
uma árvore $T = (V, E)$, desenvolva um algoritmo que determina a
cardinalidade da menor cobertura de vértices de $T$.  Seu algoritmo
deve ter custo de tempo $O(|V|)$.  Dica: cada vértice $v \in V$ tem
vizinhos $N(v) \subseteq V \setminus \{v\}$, e remover $v$ de $T$
produz $|N(v)|$ árvores.

\vspace{\baselineskip}

\textbf{Questão 12}

Considere um conjunto $A = \{a_1, a_2, \dots, a_n\}$ de inteiros.
Dado um inteiro $t$, desenvolva um algoritmo que decide se o somatório
de algum subconjunto de $A$ é maior que $t$.  Seu algoritmo deve ter
custo de tempo $O(nt)$.

\vspace{\baselineskip}

\textbf{Questão 13}

Considere um conjunto $A = \{a_1, a_2, \dots, a_n\}$ de inteiros.
Dado um inteiro $t$, desenvolva um algoritmo que decide se o somatório
de algum subconjunto de $A$ é exatamente $t$.  Seu algoritmo deve ter
custo de tempo $O(nt)$.

\vspace{\baselineskip}

\textbf{Questão 14}

Tome um conjunto de inteiros $A = \{a_1, a_2, \dots, a_n\}$.
Desenvolva um algoritmo para determinar se existe uma partição
$A = B \cup C, B \cap C = \emptyset$, de forma que
$\sum_{a \in B} a = \sum_{a \in C} a$.  Seu algoritmo deve ser
polinomial em $n$ e $\sum_{i = 1}^n a_i$.

\vspace{\baselineskip}

\textbf{Questão 15}

Tome um conjunto de inteiros $A = \{a_1, a_2, \dots, a_n\}$.
Desenvolva um algoritmo para particionar $A$ em
$A = B \cup C, B \cap C = \emptyset$, de forma que
$|\sum_{a \in B} a - \sum_{a \in C} a|$ seja mínimo.  Seu algoritmo
deve ser polinomial em $n$ e $\sum_{i = 1}^n a_i$.

\vspace{\baselineskip}

\textbf{Questão 16}

Tome um conjunto de inteiros $A = \{a_1, a_2, \dots, a_n\}$.
Desenvolva um algoritmo para particionar $A$ em $A = B \cup C \cup D$,
onde o somatório das três partes é igual entre si.  Seu algoritmo deve
ser polinomial em $n$ e $\sum_{i = 1}^n a_i$.

\vspace{\baselineskip}

\textbf{Questão 17}

Tome $D = (V, A)$ um digrafo sem ciclos negativos,
$w : A \to \mathbb{R}$, e $s \in V$.  Para cada $v \in V$, seja $n_v$
o menor número de arcos em um caminho mais curto de $s$ a $v$.  Seja
$m = \max\{n_v : v \in V\}$.  Altere o algoritmo de Bellman-Ford de
forma que o conteúdo do primeiro laço precise ser executado apenas
$m + 1$ vezes, mesmo que $m$ seja um valor desconhecido.  Argumente.
Dica: talvez seja preciso mudar o laço de ``Para'' para ``Enquanto''.

\vspace{\baselineskip}

\textbf{Questão 18}

Desenvolva um algoritmo eficiente (tempo polinomial) para encontrar o
número total de caminhos em um dag (\emph{directed acyclic graph}).
Dica: todo caminho termina em algum vértice.

\vspace{\baselineskip}

\textbf{Questão 19}

Tome um digrafo $D = (V, A)$ e uma função de confiabilidade
$p : A \to [0, 1]$.  Se $D$ está representando um sistema de
comunicação, por exemplo, então $p$ indica a probabilidade das
conexões (representadas pelos arcos de $D$) não falharem.
Considerando que essas probabilidades são independentes, desenvolva um
algoritmo para encontrar o caminho mais confiável de $u$ a $v$, dados
como parte da entrada.

\vspace{\baselineskip}

\textbf{Questão 20}

É possível utilizar a saída do algoritmo de Floyd-Warshall para
determinar a presença de ciclos com peso negativo no grafo de entrada?
Dica: qual a menor distância de um vértice para si mesmo?

\vspace{\baselineskip}

\textbf{Questão 21}

Tome um conjunto finito $S$ e um inteiro $k$.  Dado
$\mathcal{I} = \{C \subseteq S : |C| \leq k\}$, prove que
$M = (S, \mathcal{I})$ é um matróide.  Que problema envolvendo $M$ e
uma função de peso $w : S \to \mathbb{R}_+$, estritamente positiva, um
algoritmo guloso seria capaz de resolver?

\vspace{\baselineskip}

\textbf{Questão 22}

Tome $A \in \mathbb{R}^{m \times n}$.  Seja $S$ o conjunto de colunas
de $A$ e
$\mathcal{I} = \{C \subseteq S : C \text{ é linearmente
  independente}\}$.  Prove que $M = (S, \mathcal{I})$ é um matróide.

\vspace{\baselineskip}

\textbf{Questão 23}

Seja $M = (S, \mathcal{I})$ um matróide.  Tome
$\mathcal{B} \subseteq \mathcal{I}$ a família de subconjuntos
independentes maximais de $M$.  Definimos
$\mathcal{I}' = \{C \subseteq S \,|\, \exists\, B \in \mathcal{B} : B
\subseteq S \setminus C\}$.  Prove que $M' = (S, \mathcal{I}')$ é um
matróide.

\vspace{\baselineskip}

\textbf{Questão 24}

Dado um matróide $M = (S, \mathcal{I})$ e uma função de peso
$w : S \to \mathbb{R}_+$ estritamente positiva, considere o problema
$P$ de encontrar um conjunto independente maximal de $M$ com peso
mínimo.  Descreva uma função de peso $w' : S \to \mathbb{R}_+$,
estritamente positiva, tal que um conjunto independente de $M$ com
peso máximo, de acordo com $w'$, seja uma solução para $P$.

\vspace{\baselineskip}

\textbf{Questão 25}

Tome um grafo $G = (V, E)$ e uma função de peso
$w : E \to \mathbb{R}_+$.  Um corte de $G$ é um conjunto
$E' \subseteq E$ tal que $G' = (V, E \setminus E')$ é desconexo.
Sabendo que uma árvore geradora de $G$ tem pelo menos uma aresta de
cada corte de $G$, desenvolva um algoritmo, por indução, para
encontrar uma árvore geradora de peso mínimo de $G$.  Dica: a remoção
de um corte minimal de $G$ o divide em dois subgrafos conexos menores,
digamos $G_1$ e $G_2$; como são menores, uma hipótese nos daria
árvores geradoras de $G_1$ e $G_2$; o que resta para construir uma
árvore geradora de $G$?  Outra dica: as arestas incidentes em um
vértice formam um corte de $G$; de forma generalizada, as arestas
incidentes em um conjunto de vértices de $G$ formam um corte de $G$.

\end{document}
%%% Local Variables:
%%% mode: latex
%%% TeX-master: t
%%% End:
