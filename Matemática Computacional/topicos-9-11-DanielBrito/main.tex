\documentclass[]{article}

\usepackage[brazil]{babel}
\usepackage[utf8]{inputenc}

\usepackage{amsmath}
\usepackage{amsfonts}
\usepackage{amsthm}
\usepackage[shortlabels]{enumitem}
\usepackage[dvipsnames]{xcolor}

\usepackage{hyperref}

\begin{document}
75) (Ida) Assuma que $\max (-c)^Tx > (-c)^T\overline{x}$. Uma vez que isto é definido, existe alguma outra valoração para $x$, que não é o $\overline{x}$, digamos um $\tilde{x}$, de tal forma que $(-c)^T\tilde{x} > (-c)^T\overline{x}$. Sabendo disso, quando dissemos que $\min c^Tx = c^T\overline{x}$ era nossa hipótese, estávamos admitindo que $\overline{x}$ era solução ótima, e agora estamos dizendo, no contexto da contradição, que $\tilde{x}$ é tal que $(-c)^T\tilde{x} > (-c)^T\overline{x}$. Entretanto, se multiplicarmos a inequação por -1, obteremos um $\tilde{x}$ menor que $\overline{x}$. Absurdo. Logo, $\min c^Tx = c^T\overline{x}$.

\vspace{0.2cm}

(Volta) De forma análoga, assuma que $\min c^Tx < c^T\overline{x}$. Uma vez que isto é definido, existe alguma outra valoração para $x$, que não é o $\overline{x}$, digamos $\hat{x}$, de tal forma que $c^T\hat{x} < c^T\overline{x}$. Sabendo disso, quando dissemos que $\max (-c)^Tx = (-c)^T\overline{x}$ era nossa hipótese, estávamos admitindo que $\overline{x}$ era solução ótima, e agora estamos dizendo, no contexto da contradição, que $\hat{x}$ é tal que $c^T\hat{x} < c^T\overline{x}$. Entretanto, se multiplicarmos a inequação por -1, obteremos um $\hat{x}$ maior que o $\overline{x}$. Absurdo. Logo, $\max (-c)^Tx = (-c)^T\overline{x}$.

\vspace{0.2cm}

Portanto, provamos que $\min c^Tx = c^T\overline{x}$ se, e somente se, $\max (-c)^Tx = (-c)^T\overline{x}$.

\vspace{0.5cm}

78) Para provar que $X \subseteq conv(X)$, precisamos mostrar que qualquer ponto de $X$ está em $conv(X)$.

Caso base ($n=1$). $X = conv(X) = x_1$.

Hipótese indutiva ($k<n$). Todo conjunto $X'$ com $|X'|<n$ pontos está em $conv(X)$.

Considere um conjunto de pontos $X = \{x_1, ..., x_n\}$.

Pela hipótese indutiva, um ponto $\overline{x} = a'_1 x_1 + ... + a'_{n-1} x_{n-1}$ está em $conv(X')$ desde que $a'_i \geq 0$, $\sum a' = 1$.

Como $a_1 x_1 + ... + a_{n-1} x_{n-1} = 1 - a_n x_n$, escolhemos $a'_i = \frac{a_i}{1 - a_i}$.

Sabemos que $\overline{x} \in conv(X') \subset conv(X)$ e $x_n \in conv(X)$.

Como $\overline{x} \in conv(X')$ e $x_n \in conv(X)$, o segmento entre $\overline{x}$ e $x_n$ está em $conv(X)$, porque $conv(X)$ é convexo.

Finalmente, o segmento formado por $\overline{x}$ e $x_n$ é dado por:

\begin{align*}
(1 - a_n) = (\frac{a_1}{1-a_n} x_1 + ... + \frac{a_{n-1}}{1-a_n} x_{n-1}) + a_n x_n = a_1 x_1 + ... + a_n x_n
\end{align*}

O que mostra que $a_1 x_1 + ... + a_n x_n$ pertence a $conv(X)$.

\vspace{0.5cm}

80) Após adicionarmos as variáveis de folga, temos:

\begin{align*}
    \max        &\quad Z = 2x_1 - x_2 + 2x_3 + 0S_1 + 0S_2 + 0S_3 \\
    \text{s.t}  &\quad 2x_1 + x_2 + S_1 = 10 \\
                &\quad x_1 + 2_x2 - 2x_3 + S_2 = 20 \\
                &\quad x_2 + 2x_3 + S_3 = 5 \\
                &\quad x_1, x_2, x_3, S_1, S_2, S_3 \geq 0
\end{align*}

\begin{center}
 \begin{tabular}{| c | c | c | c | c | c | c | c | c | c |} 
 \hline
 Iter1 &   & $C_j$ & 2 & -1 & 2 & 0 & 0 & 0 &  \\ [0.5ex] 
 \hline
 B & $C_B$ & $X_B$ & \textcolor{ForestGreen}{$x_1$} & $x_2$ & $x_3$ & $S_1$ & $S_2$ & $S_3$ & $\frac{X_B}{x_1}$ \\ 
 \hline
 \textcolor{red}{$S_1$} & 0 & 10 & \textcolor{blue}{2} & 1 & 0 & 1 & 0 & 0 & \textcolor{red}{$\frac{10}{2}=5$} \\
 \hline
 $S_2$ & 0 & 20 & 1 & 2 & -2 & 0 & 1 & 0 & $\frac{20}{1}=20$ \\
 \hline
 $S_3$ & 0 & 5 & 0 & 1 & 2 & 0 & 0 & 1 & - \\
 \hline
 $Z=0$ &   & $Z_j$ & 0 & 0 & 0 & 0 & 0 & 0 &   \\
 \hline
  &   & $Z_j - C_j$ & \textcolor{ForestGreen}{-2} & 1 & -2 & 0 & 0 & 0 &   \\
 \hline
\end{tabular}
\end{center}

\begin{align*}
    L_1 \leftarrow \frac{L_1}{2} \\
    L_2 \leftarrow L_2 - L_1 \\
    L_3 \leftarrow L_3
\end{align*}

\begin{center}
 \begin{tabular}{| c | c | c | c | c | c | c | c | c | c |} 
 \hline
 Iter2 &   & $C_j$ & 2 & -1 & 2 & 0 & 0 & 0 &  \\ [0.5ex] 
 \hline
 B & $C_B$ & $X_B$ & $x_1$ & $x_2$ & \textcolor{ForestGreen}{$x_3$} & $S_1$ & $S_2$ & $S_3$ & $\frac{X_B}{x_3}$ \\ 
 \hline
 $x_1$ & 2 & 5 & 1 & 0.5 & 0 & 0.5 & 0 & 0 & - \\
 \hline
 $S_2$ & 0 & 15 & 0 & 1.5 & -2 & -0.5 & 1 & 0 & - \\
 \hline
 \textcolor{red}{$S_3$} & 0 & 5 & 0 & 1 & \textcolor{blue}{2} & 0 & 0 & 1 & \textcolor{red}{$\frac{5}{2}=2.5$} \\
 \hline
 $Z=10$ &   & $Z_j$ & 2 & 1 & 0 & 1 & 0 & 0 &   \\
 \hline
  &   & $Z_j - C_j$ & 0 & 2 & \textcolor{ForestGreen}{-2} & 1 & 0 & 0 &   \\
 \hline
\end{tabular}
\end{center}

\begin{align*}
    L_3 \leftarrow \frac{L_3}{2} \\
    L_1 \leftarrow L_1 \\
    L_2 \leftarrow L_2 + 2L_3
\end{align*}

\begin{center}
 \begin{tabular}{| c | c | c | c | c | c | c | c | c | c |} 
 \hline
 Iter3 &   & $C_j$ & 2 & -1 & 2 & 0 & 0 & 0 &  \\ [0.5ex] 
 \hline
 B & $C_B$ & $X_B$ & $x_1$ & $x_2$ & $x_3$ & $S_1$ & $S_2$ & $S_3$ &  \\ 
 \hline
 $x_1$ & 2 & 5 & 1 & 0.5 & 0 & 0.5 & 0 & 0 &  \\
 \hline
 $S_2$ & 0 & 20 & 0 & 2.5 & 0 & -0.5 & 1 & 1 &  \\
 \hline
 $x_3$ & 2 & 2.5 & 0 & 0.5 & 1 & 0 & 0 & 0.5 &  \\
 \hline
 $Z=15$ &   & $Z_j$ & 2 & 2 & 2 & 1 & 0 & 1 &   \\
 \hline
  &   & $Z_j - C_j$ & 0 & 3 & 0 & 1 & 0 & 1 &   \\
 \hline
\end{tabular}
\end{center}

Como todos os valores $Z_j - C_j \geq 0$, temos:

\begin{align*}
    x_1=5, x_2=0, x_3=2.5 \\
    \max Z = 15
\end{align*}

\vspace{0.5cm}

81) Após adicionarmos as variáveis de folga, temos:

\begin{align*}
    \max        &\quad Z = 2x_1 - x_2 + 2x_3 + 0S_1 + 0S_2 \\
    \text{s.t}  &\quad x_1 - x_2 + S_1 = 10 \\
                &\quad 2x_1 - x2 + S_2 = 40 \\
                &\quad x_1, x_2, S_1, S_2 \geq 0
\end{align*}

\begin{center}
 \begin{tabular}{| c | c | c | c | c | c | c | c |} 
 \hline
 Iter1 &   & $C_j$ & 2 & 1 & 0 & 0 &  \\ [0.5ex] 
 \hline
 B & $C_B$ & $X_B$ & \textcolor{ForestGreen}{$x_1$} & $x_2$ & $S_1$ & $S_2$ & $\frac{X_B}{x_1}$ \\ 
 \hline
 \textcolor{red}{$S_1$} & 0 & 10 & \textcolor{blue}{1} & -1 & 1 & 0 & \textcolor{red}{$\frac{10}{1}=10$} \\
 \hline
 $S_2$ & 0 & 40 & 2 & -1 & 0 & 1 & $\frac{40}{2}=20$  \\
 \hline
 $Z=0$ &   & $Z_j$ & 0 & 0 & 0 & 0 &   \\
 \hline
  &   & $Z_j - C_j$ & \textcolor{ForestGreen}{-2} & -1 & 0 & 0 &  \\
 \hline
\end{tabular}
\end{center}

\begin{align*}
    L_1 \leftarrow L_1 \\
    L_2 \leftarrow L_2 - 2L_1
\end{align*}

\begin{center}
 \begin{tabular}{| c | c | c | c | c | c | c | c |} 
 \hline
 Iter2 &   & $C_j$ & 2 & 1 & 0 & 0 &  \\ [0.5ex] 
 \hline
 B & $C_B$ & $X_B$ & $x_1$ & \textcolor{ForestGreen}{$x_2$} & $S_1$ & $S_2$ & $\frac{X_B}{x_2}$ \\ 
 \hline
 $x_1$ & 2 & 10 & 1 & -1 & 1 & 0 & -\\
 \hline
 \textcolor{red}{$S_2$} & 0 & 20 & 0 & \textcolor{blue}{1} & -2 & 1 & \textcolor{red}{$\frac{20}{1}=20$}  \\
 \hline
 $Z=20$ &   & $Z_j$ & 2 & -2 & 2 & 0 &   \\
 \hline
  &   & $Z_j - C_j$ & 0 & \textcolor{ForestGreen}{-3} & 2 & 0 &  \\
 \hline
\end{tabular}
\end{center}

\begin{align*}
    L_2 \leftarrow L_2 \\
    L_1 \leftarrow L_1 + L_2
\end{align*}

\begin{center}
 \begin{tabular}{| c | c | c | c | c | c | c | c |} 
 \hline
 Iter3 &   & $C_j$ & 2 & 1 & 0 & 0 &  \\ [0.5ex] 
 \hline
 B & $C_B$ & $X_B$ & $x_1$ & $x_2$ & \textcolor{ForestGreen}{$S_1$} & $S_2$ & $\frac{X_B}{S_1}$ \\ 
 \hline
 $x_1$ & 2 & 30 & 1 & 0 & -1 & 1 & -\\
 \hline
 $x_2$ & 1 & 20 & 0 & 1 & -2 & 1 & -\\
 \hline
 $Z=80$ &   & $Z_j$ & 2 & 1 & -4 & 3 &   \\
 \hline
  &   & $Z_j - C_j$ & 0 & 0 & \textcolor{ForestGreen}{-4} & 3 &  \\
 \hline
\end{tabular}
\end{center}

Como todos os coeficientes em $S_1$ são negativos ou nulos, a solução para o problema é ilimitada.

\vspace{0.5cm}

82) Após adicionarmos as variáveis de folga, temos:

\begin{align*}
    \max        &\quad Z = 4x_1 + 14x_2 + 0S_1 + 0S_2 \\
    \text{s.t}  &\quad 2x_1 + 7x_2 + S_1 = 21 \\
                &\quad 7x_1 + 2x2 + S_2 = 21 \\
                &\quad x_1, x_2, S_1, S_2 \geq 0
\end{align*}

\begin{center}
 \begin{tabular}{| c | c | c | c | c | c | c | c |} 
 \hline
 Iter1 &   & $C_j$ & 4 & 14 & 0 & 0 &  \\ [0.5ex] 
 \hline
 B & $C_B$ & $X_B$ & $x_1$ & \textcolor{ForestGreen}{$x_2$} & $S_1$ & $S_2$ & $\frac{X_B}{x_2}$ \\ 
 \hline
 \textcolor{red}{$S_1$} & 0 & 21 & 2 & \textcolor{blue}{7} & 1 & 0 & \textcolor{red}{$\frac{21}{7}=3$} \\
 \hline
 $S_2$ & 0 & 21 & 7 & 2 & 0 & 1 & $\frac{21}{2}=10.5$\\
 \hline
 $Z=0$ &   & $Z_j$ & 0 & 0 & 0 & 0 &   \\
 \hline
  &   & $Z_j - C_j$ & -4 & \textcolor{ForestGreen}{-14} & 0 & 0 &  \\
 \hline
\end{tabular}
\end{center}

\begin{align*}
    L_1 \leftarrow \frac{L_1}{7} \\
    L_2 \leftarrow L_2 + 2L_1
\end{align*}

\begin{center}
 \begin{tabular}{| c | c | c | c | c | c | c | c |} 
 \hline
 Iter2 &   & $C_j$ & 4 & 14 & 0 & 0 &  \\ [0.5ex] 
 \hline
 B & $C_B$ & $X_B$ & $x_1$ & $x_2$ & $S_1$ & $S_2$ &  \\ 
 \hline
 $x_2$ & 14 & 3 & 0.2857 & 1 & 0.1429 & 0 &  \\
 \hline
 $S_2$ & 0 & 15 & 6.4286 & 0 & -0.2857 & 1 &  \\
 \hline
 $Z=42$ &   & $Z_j$ & 4 & 14 & 2 & 0 &   \\
 \hline
  &   & $Z_j - C_j$ & 0 & 0 & 2 & 0 &  \\
 \hline
\end{tabular}
\end{center}

Como todos os valores $Z_j - C_j \geq 0$, temos:

\begin{align*}
    x_1=0, x_2=3 \\
    \max Z = 42
\end{align*}

Entretanto, quando o custo de uma variável não-básica é nulo, podemos ter outras soluções ótimas. Logo:

\begin{center}
 \begin{tabular}{| c | c | c | c | c | c | c | c |} 
 \hline
 Iter2 &   & $C_j$ & 4 & 14 & 0 & 0 &  \\ [0.5ex] 
 \hline
 B & $C_B$ & $X_B$ & \textcolor{ForestGreen}{$x_1$} & $x_2$ & $S_1$ & $S_2$ & $\frac{X_B}{x_1}$ \\ 
 \hline
 $x_2$ & 14 & 3 & 0.2857 & 1 & 0.1429 & 0 & $\frac{3}{0.2857}=10.5$ \\
 \hline
 \textcolor{red}{$S_2$} & 0 & 15 & \textcolor{blue}{6.4286} & 0 & -0.2857 & 1 & \textcolor{red}{$\frac{15}{6.4286}=2.3333$} \\
 \hline
 $Z=42$ &   & $Z_j$ & 4 & 14 & 2 & 0 &   \\
 \hline
  &   & $Z_j - C_j$ & \textcolor{ForestGreen}{0} & 0 & 2 & 0 &  \\
 \hline
\end{tabular}
\end{center}

\begin{align*}
    L_2 \leftarrow \frac{L_2}{6.4286} \\
    L_1 \leftarrow L_1 - 0.2857L_2
\end{align*}

\begin{center}
 \begin{tabular}{| c | c | c | c | c | c | c | c |} 
 \hline
 Iter3 &   & $C_j$ & 4 & 14 & 0 & 0 &  \\ [0.5ex] 
 \hline
 B & $C_B$ & $X_B$ & $x_1$ & $x_2$ & $S_1$ & $S_2$ &  \\ 
 \hline
 $x_2$ & 14 & 2.3333 & 0 & 1 & 0.1556 & -0.0444 &  \\
 \hline
 $x_1$ & 4 & 2.3333 & 1 & 0 & -0.0444 & 0.1556 &  \\
 \hline
 $Z=42$ &   & $Z_j$ & 4 & 14 & 2 & 0 &   \\
 \hline
  &   & $Z_j - C_j$ & 0 & 0 & 2 & 0 &  \\
 \hline
\end{tabular}
\end{center}

Como todos os valores $Z_j - C_j \geq 0$, temos:

\begin{align*}
    x_1=2.3333, x_2=2.3333 \\
    \max Z = 42
\end{align*}

\end{document}